\documentclass[a4paper]{article}
\usepackage{amsmath, amssymb}
\usepackage[margin=2cm]{geometry}

\begin{document}
\setlength{\parindent}{0pt}
\setlength{\parskip}{1em}

\section{Introduction}
A \textit{graph} $G$ consists of a set of \textit{vertices} $V(G)$ and a set of \textit{edges} $E(G)$ 

Each edge is associated with one or two vertices, its \textit{ends}

An edge is a \textit{loop} if the ends are the same, a \textit{link} if the ends are different

Graphs are assumed to be \textit{finite} - both $V(G)$ and $E(G)$ are finite sets

A \textit{path} is a sequence $(a_0, A_1, a_1, A_2, a_2, \ldots, A_n, a_n)$ of vertices $a_i$ and edges $A_j$ such that:
\begin{enumerate}
\item If $1 \leq i \leq n$ the ends of $A_i$ are $a_{i-1}$ and $a_i$
\item If $1 \leq i \leq n$ then $a_{i-1} = a_i$ iff $A_i$ is a loop
\end{enumerate}

If all terms in path are distinct, then the path is \textit{simple}

If all terms are distinct except that $a_0 = a_n$, then the path is \textit{circular}

$x,y \in V(G)$ are \textit{connected} if there is a path from $x$ to $y$ in $G$

Connectedness is an equivalence relation on $V(G)$ so that if $V(G) \neq \varnothing$ then it can be partitioned into disjoint non-null subsets $V_1, \ldots, V_k$ such that two vertices of $G$ are connected iff they are in the same $V_i$

The $G[V_i]$ are the \textit{connected components} of $G$, they are edge- and vertex-disjoint and cover $G$

The number of components of $G$ is denoted $p_0(G)$

A graph is \textit{connected} iff $p_0(G) = 0$ or $1$, the first case arising if $G$ is the empty graph

A connected graph with no circular path is a tree

$\alpha_0(G) = |G|$ and $\alpha_1(G) = e(G)$

Let $Q_n$ be a finite set of $n > 0$ elements

$f:V(G) \to Q_n$ is an \textit{n-colouring} of $G$ if each edge $xy$ of $G$ has $f(x) \neq f(y)$

The number of $n$-colourings of $G$ wrt $Q_n$ is denoted by $P(G,n)$

If $V(G) = 0$ then we say $P(G,n) = 1$, also $P(G,n) = 0$ if $G$ contains a loop

When $G$ is loopless, $P(G,n)$ is a polynomial in $n$ of degree $|G|$

For planar graphs, $P(G,n)$ is called the \textit{chromatic polynomial} of $G$
\begin{align}
P(G,n) = \sum_S (-1)^{e(S)} n^{p_0(S)} \tag{summing over spanning subgraphs $S$ of $G$}
\end{align}

An \textit{orientation} of $G$ distinguishes one end of each edge $A$ as positive, $p(A)$, and one as negative, $q(A)$

If $A$ is a loop, then $p(A) = q(A)$ otherwise $p(A) \neq q(A)$

If $a \in V(G)$ and $A \in E(G)$ then $\eta(A,a) = 0$ if $A$ is a loop or $a$ is not an end of $A$.\\
Otherwise, $\eta(A,a) = 1$ or $-1$ depending as whether $a$ is the positive or negative end of $A$

A mapping $f$ of $V(G)$ or $E(G)$ into $Q_n$ is a \textit{0-chain} or \textit{1-chain} respectively \textit{on} $G$ \textit{over} $Q_n$

If $V(G) = \varnothing$ then there is just one $0$-chain on $G$ over $Q_n$

If $E(G) = \varnothing$ then there is just one $1$-chain on $G$ over $Q_n$

If $h$ is a $0$-chain on $G$ over $Q_n$ its \textit{coboundary}, $\delta h$ is the $1$-chain on $G$ over $Q_n$ satisfying
\begin{align}
(\delta h)(A) = \sum_a \eta(A,a)h(a) \tag{2}
\end{align}
for each $A \in E(G)$, equivalently
\begin{align}
(\delta h)(A) = h(p(A)) - h(q(A)) \tag{2a}
\end{align}

If $g$ is a $1$-chain, its \textit{boundary} $\delta g$ is the $0$-chain satisfying
\begin{align}
(\delta g)(a) = \sum_A \eta(A,a)g(A) \tag{3}
\end{align}
for each $a \in V(G)$

We call $g$ a \textit{1-cycle} on $G$ over $Q_n$ if $\delta g \equiv \boldsymbol{0}$

\section{Colour-coboundaries and colour-cycles}

A \textit{colour-coboundary} or \textit{colour-cycle} on $G$ over $Q_n$ is a $1$-chain $g$ on $G$ over $Q_n$ which is a coboundary or a 1-cycle respectivly and which satisfies $g(A) \neq 0$ for each $A \in E(G)$

The number of colour-coboundaries of $G$ over $Q_n$ is denoted $\theta(G,n)$

The number of colour cycles on $G$ over $Q_n$ is denoted $\phi(G,n)$

$\theta(G,n)$ and $\phi(G,n)$ are independent of orientation

If $e(G) = 0$ then we say $\theta(G,n) = \phi(G,n) = 1$

By (2a), the colour-coboundaries on $G$ over $Q_n$ are the coboundaries of the $n$-colourings of $G$

Also, $\delta h_1 = \delta h_2$ for 0-chains $h_1, h_2$ iff $h_1(a) - h_2(a)$ is constant in each component of $G$, for all $A \in E(G)$:
\begin{align*}
h_1(p(A)) - h_1(q(A)) = h_2(p(A)) - h_2(q(A)) \iff\ h_1(p(A)) - h_2(p(A)) = h_1(q(A)) - h_2(q(A))
\end{align*}




\end{document}