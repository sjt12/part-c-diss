\documentclass[a4paper]{article}
\usepackage{amsmath, amssymb}
\usepackage[margin=2cm]{geometry}

\openup 0.75em

\begin{document}
\setlength{\parindent}{0pt}

\section{Introduction}

Let $E$ be a finite set, a \textit{chain} on $E$ is a mapping $f$ of $E$ into $\mathbb{F}_2$

If $x \in E$ then $fx$ is the \textit{coefficient} of $x$ in $f$

Chains can be added to give new chains by adding coefficients for each $x \in E$

Thus chains form an \textit{additive group}

Any subgroup of this group is called a \textit{chain-group} on $E$

The zero element in a chain group is the chain in which each coefficient is zero, this is the \textit{zero chain}

Let $N$ be a chain-group on $E$, the elements of $E$ are \textit{cells} of $N$

A cell $x$ is \textit{filled} if $fx = 1$ for some chain $f$ of $N$ and \textit{empty} otherwise

$N$ is a \textit{full chain group} if it has no empty cells

A colouring of $N$ is a pair $\{f,g\}$ of chains of $N$ such that for each $x \in E$ either $fx = 1$ or $gx = 1$

$N$ is \textit{chromatic} if it has a colouring, and \textit{achromatic} otherwise

A chromatic chain-group is necessarily full


\section{The Cycles and Coboundaries of a Graph}

An edge is a \textit{loop} if the ends are the same, a \textit{link} if the ends are different

Graphs are assumed to be \textit{finite} - both $V(G)$ and $E(G)$ are finite sets

A \textit{path} is a sequence $(a_0, A_1, a_1, A_2, a_2, \ldots, A_n, a_n)$ of vertices $a_i$ and edges $A_j$ such that:
\begin{enumerate}
\item If $1 \leq i \leq n$ the ends of $A_i$ are $a_{i-1}$ and $a_i$
\item If $1 \leq i \leq n$ then $a_{i-1} = a_i$ iff $A_i$ is a loop
\end{enumerate}

If all terms in path are distinct, then the path is \textit{simple}

If all terms are distinct except that $a_0 = a_n$, then the path is \textit{circular}







\end{document}