\documentclass[11pt]{article}

\usepackage{amssymb}
\usepackage{amssymb,html,url}
\usepackage[hidelinks]{hyperref}


\setlength{\topmargin}{-0.8in}
\setlength{\oddsidemargin}{0.1in}
\setlength{\textheight}{10in}
\setlength{\textwidth}{6in}
\setlength{\parskip}{0.4\baselineskip}
\setlength{\parindent}{0in}

\newcommand{\h}{\hspace{0.5in}}
\newcommand{\vs}{\vspace{0.2in}}
\newcommand{\no}{\noindent}
\newcommand{\checkbox}{\makebox[0pt][l]{$\square$}\raisebox{.15ex}{\hspace{0.1em}$\checkmark$}}

\pagestyle{empty}


\begin{document}

\begin{center}
\textbf{Oxford University: Mathematical Institute}

\textbf{FORM OF APPLICATION TO THE PROJECTS COMMITTEE}

\textbf{READ THE RELEVANT EXAMINATION DECREES AND REGULATIONS FIRST}



\end{center}
\bigskip

\bigskip

Your name (block capitals) \dots\dots\ {\sc Stephen Thatcher} \dotfill

Your supervisor's name (block capitals) \dots\dots\ {\sc Prof. Alex Scott} \dotfill

The title of your project (block capitals) \dots\dots\ {\sc Tutte's Flow Conjectures} \dotfill

Your email address \dots\dots\ {\sc stephen.thatcher@merton.ox.ac.uk } \dotfill

Your college (block capitals) \dots\dots\ {\sc Merton College} \dotfill

Your course \dots\dots\ {\sc Master of Mathematics and Philosophy} \dotfill


\medskip



\textbf{Application for approval of topic of: (tick one box)}


\bigskip

\textbf{For 3rd year students}


{\huge $\Box$} BEE Mathematical Extended Essay\\
{\huge $\Box$} BOE Other Mathematical Extended Essay

\bigskip

\textbf{For 4th year students}

{\huge \checkbox} CCD Mathematical Dissertation 



{\huge $\Box$} COD Other Mathematical Dissertation 

 



\textbf{For 4th year students only}

\begin{enumerate}
\item[(a)] Please specify the courses you offered for examination in Part B 

{\sc B1.1 Logic, B1.2 Set Theory, B3.1 Galois Theory,} \dotfill

{\sc B3.4 Algebraic Number Theory, B3.5 Topology and Groups,} \dotfill

{\sc B8.5 Graph Theory} \dotfill

\item[(b)] Did you offer an extended essay?  If so, what was its title? \dotfill\ {\sc n/a} \dotfill

.\dotfill

\end{enumerate}

\newpage

\textbf{Insert here a typed, brief (at least 100 words) description of the project,\linebreak including references.} 

\begin{quote}
For $G = (V,E)$ a directed multigraph and $H$ an abelian group, a map $f: E \to H$ is an $H$-\textit{flow} if $f(e) \neq 0$ for all $e \in E$ and $f$ obeys Kirchhoff's law: for all $v \in V$
$$\sum_{e \in \delta^+(v)} f(e) = \sum_{e \in \delta^-(v)} f(e)$$
If $H = \mathbb{Z}$, the group of integers under addition, and $k$ is a positive integer such that $-k < f(e) < k$ for every edge $e$, we say that $f$ is a $k$-flow. 

W. T. Tutte made a series of conjectures asserting certain weak conditions under which a multigraph $G$ has a 3-, 4- or 5-flow.  We will present the statements of these alongside a number of results about group-valued and $k$-flows including sufficient conditions for low $k$-flows and the duality between flow and colouring problems. As an example of this we shall demonstrate that the theorem of Grotzsch that \emph{every triangle free planar graph is 3-colourable} is a dual to the 3-flow conjecture with an added assumption of planarity. Finally we will give an exposition of Thomassen's recent proof of a weakening of the 3-flow conjecture and discuss subsequent work.



%In this essay we will introduce the concepts of group valued flows and $k$-flows - flows with . in order to facilitate discussion of Tutte's 3-, 4-, and 5-flow conjectures. . An account of the duality between flows and colouring will be given including a demonstration that the theorem of Grotzsch that every triangle free planar graph is 3-colourable is dual to the 3-flow conjecture with an added assumption of planarity. The essay will then present an exposition of the proof by Thomassen of a weakened form of the 3-flow conjecture, as proposed by Jaeger. 

\nocite{thom}
\nocite{GT}
\nocite{lova}
\nocite{T1}
\nocite{T2}

\bibliographystyle{plain}
\bibliography{biblio}
\end{quote}



\newpage

\textbf{Obtain here a statement of approval and a recommendation of three possible assessors from the supervisor of the
  project.}

\smallskip

Approval: (please continue on a separate sheet if necessary).
[Note that the Committee will be very grateful if,
before giving your approval,
you check that the proposal does indeed satisfy the requirements
in the
\htmladdnormallink{`Projects Guidance Notes'}{http://www.maths.ox.ac.uk/current-students/undergraduates/projects/}\\
(\htmladdnormallink{http://www.maths.ox.ac.uk/current-students/undergraduates/projects/}{http://www.maths.ox.ac.uk/current-students/undergraduates/projects/}).]


\vfill




\vspace{3.0in}
Recommended Assessors:\dotfill

\medskip
Signed (Supervisor): \dotfill

Date: \dotfill



\medskip
Signature of College Tutor: \dotfill

 Date: \dotfill
\medskip

\textbf{Return the form in a sealed envelope to Mrs Helen Lowe, Deputy Academic Administrator in the Mathematical
  Institute no later than 12noon on Friday of 0th week of Michaelmas Term.
  Keep a copy yourself.}\\

\textbf{All applications are acknowledged. If you have not received an email confirming the receipt of you application within four working days of getting it to the Mathematical Institute, please contact Mrs Lowe by emailing helen.lowe@maths.ox.ac.uk.}

\vspace*{25pt}






\end{document}